\documentclass[a4paper,10pt]{article}
\usepackage[utf8]{inputenc}
\usepackage{expl3}
\usepackage{pgfplots}
\pgfplotsset{compat=1.12}


%opening
\title{Character distribution}
\author{Amadou Kountché Djibrilla}

\begin{document}

\maketitle



\section{Introduction}
The $\chi^2$ test can be use to assess the independence of two random variables or to test the hypothesis
that an individual variable is drown from a distribution. 
In the case of intrusion detection, we are going to use the second case. 


\section{Definitons}
Given $v$ independent variables, each normally distributed with mean $u_i$ and $\sigma^2_i$, then :
\begin{equation}
 \chi^2 = \sum^{v}_{i=1} (\frac{(x_i - \mu_i)^2}{\sigma^{2}_{i}}
\end{equation}

Ideally, given the random fluctuations of the values of $\xi$ about their mean value $\mu_i$, each term in the sim will be of order of unity, 
hence if $u_i$ and $\sigma_i$ are choosen correctly, the $\chi^2$ value will be approximatlly eaqual to $v$.

If this is the case, it can be concluded that $u_i$ and $\sigma_i$ describe well the data, the we can not reject the hypothesis. 


If $\chi^2$ is greater than v, and we have correctly estimated the value of $\sigma_i$, we may possibly conclude that our data are not well described by our hypothesized set
of the $u_i$. 

This is the general idea of $\chi^2$ test. 


\section{The $chi^2$ distribution}

The distribution of the random variable $\chi^2$ is :
\begin{equation}
 f(\chi^2) = \frac{1}{2^{v/2} \Gamma (v/2)} e^{-\chi^2/2}(\chi^2)^{(v/2)^{-1}}
\end{equation}

where: 
\begin{itemize}
 \item $v$ is the degree of freedom
 \item $\Gamma(p)$ is the gamma function. 
 
\end{itemize}

The gamma function is defined by : 
\begin{equation}
  \Gamma (p +1)  \equiv \int^{\infty}_{0} x^p e^{-x} dx
 \end{equation}
 
 
\begin{itemize}
 \item It is a generalization of the factorial function to non-integer value of $p$;
 \item if $p$ is an, $\Gamma(p+1) = p!$
 \item in general, $\Gamma(p+1) = p\Gamma(p)$
 \item $Gamma(1/2) = \sqrt{\pi}$
\end{itemize}


\begin{figure}
 \begin{tikzpicture}
  \begin{axis}[%
    xlabel = $x$,
    ylabel = {Probability density},
    samples = 200,
    restrict y to domain = 0:0.5,
    domain = 0.01:15]
    \foreach \k in {1,...,8} {%
      \addplot+[mark={}] gnuplot[raw gnuplot] {%
        isint(x) = (int(x)==x);
        log2 = 0.693147180559945;
        chisq(x,k)=k<=0||!isint(k)?1/0:x<=0?0.0:exp((0.5*k-1.0)*log(x)-0.5*x-lgamma(0.5*k)-k*0.5*log2);
        set xrange [1.00000e-5:15.0000];
        set yrange [0.00000:0.500000];
        samples=200;
        plot chisq(x,\k)};
    \addlegendentryexpanded{$k = \k$}}
  \end{axis}
\end{tikzpicture}
\caption{Illustration of the $\chi^2$ distribution}
\end{figure}

The $\chi^2$ distribution is skewed for small values and tend toward the  normal distribution. 


\section{Using the $\chi^2$ for statistical test}

\begin{itemize}
 \item Suppose we have $N$ experimental measured quantities $x_i$,
 \item we want to known whether ther are well described by some set of hypothesized values $\mu_i$
 \item Determine the value of $\chi^2$ as described in the equation. In determining the sum, we must use estimates for the $\sigma_i$
  that are independently obtained for each $\sigma_i$. 
  
  
  We can generalise from above discussion, to say that we expect a single measured value of $\chi^2$ will have a aprobaility $\alpha$ of being greater
  than $\chi^2_{v,\alpha}$ defined by : 
  \begin{equation}
   \int^{}_{\infty} f(\chi^2) d\chi^2 = \alpha
  \end{equation}

\end{itemize}

 \begin{figure}
  
  \caption{Illustrate here the equation of $\chi^2_{v,\alpha}$. }
 \end{figure}

The following steps Illustrate how to use the test : 
\begin{enumerate}
 \item We hypothesize that our data are approprially described by our chosen function, or set of $\mu_i$. This is the hypothesis 
 we are going to test. 
 
 \item From our data sample, we calculate a sample value of $\chi^2$, along with $v$, and so determine $\chi^2/v$ (the normalized chi-squarre, or chi-square per degree of freedom) for our sample. 
 
 \item we choose a value of the significance level $\alpha$ (0.05 is a common value) anf from an appropriate table or graph, determine the corresponding
 value of $\chi^2_{v,\alpha}/v$. We then compare this value with our sample value of $chi^2/v$
 
 \item If we find that $\chi^2/v > \chi^2_{v,\alpha}$, we may conclude that either (i) the model represented by the $\mu_i$ is a valid one but thant a staistically
 improbable excursion of $\chi^2$ has occured, or (ii) that our model is so poorly choosen that an unacceptably large value of $\chi^2$ has resulted.
 
 (i) will happen with a probability $\alpha$, so if we are satisfied that (i) and (ii) are the only possibilities, (ii) will happen with a probability of $1 - \alpha$.
 
 Thus, if we find that $\chi^2/v > \chi^2_{v,\alpha}$, we are $100 \times (1 - \alpha)$ per cent confident in $rejecting$ our model.
 Note that this reasoning breaks down if there is a possibily (iii), for example if our data are not normally distributed. The theory of the chi-squarre test
 relies on the assumption that chi-square is the sum of the squares of random normal deviates, that is, that each $x_i$ is normally distributed about its mean value. 
 
However for some experiments, there may be occasional non-normal data points that are too far from the mean to be real. It is appropriate to discard data points
that are clearly outliers. 


\item If we find that $\chi^2$ is too small, that is, if $\chi^2/v < 1-\chi^2_{v,\alpha}$, we may conclude only that either (i) our model is valid but that
a staistically improbable excursion of $\chi^2$ har occured, or (ii) we have, too conservatively, over-estimated the values of $\sigma_i$ or (iii) someone has given us faudulent data, that is, data 'too good to be true'. 
A too-small value of $\chi^2$ cannot be indicative of poor model. A poor model can only increase $\chi^2$.
 
\end{enumerate}

 
\end{document} 
